\section{Section1}

\begin{frame}{good to use beamer}
    \begin{itemize}[<+-| alert@+>]
        \item everyone knowns \LaTeX{}=
        \item  Xe\LaTeX{} 
    \end{itemize}
\end{frame}

\begin{frame}
\frametitle{Sample frame title}
This is a text in second frame. 
For the sake of showing an example.

\begin{itemize}
 \item<1-> Text visible on slide 1
 \item<2-> Text visible on slide 2
 \item<3> Text visible on slide 3
 \item<4-> Text visible on slide 4
\end{itemize}
\end{frame}


\begin{frame}
 In this slide \pause

 the text will be partially visible \pause

 And finally everything will be there
\end{frame}



\begin{frame}
\frametitle{Sample frame title}

In this slide, some important text will be
\alert{highlighted} because it's important.
Please, don't abuse it.

\begin{block}{Remark}
Sample text
\end{block}

\begin{alertblock}{Important theorem}
Sample text in red box
\end{alertblock}

\begin{examples}
Sample text in green box. The title of the block is ``Examples".
\end{examples}
\end{frame}

\begin{frame}
\frametitle{Two-column slide}
\begin{columns}
\column{0.5\textwidth}
This is a text in first column. \cite{mertens_emerging_2015}
$$E=mc^2$$
\begin{itemize}
\item First item
\item Second item
\end{itemize}

\column{0.5\textwidth}
This text will be in the second column\cite{qin_recurrent_2016}
and on a second thoughts, this is a nice looking
layout in some cases.
\end{columns}
\end{frame}

\section{Codes}
\begin{frame}[fragile]
\frametitle{An Algorithm For Finding Prime Numbers.}
\begin{verbatim}
int main (void)
{
std::vector<bool> is_prime (100, true);
for (int i = 2; i < 100; i++)
if (is_prime[i])
{
std::cout << i << " ";
for (int j = i; j < 100; is_prime [j] = false, j+=i);
}
return 0;
}
\end{verbatim}
\begin{uncoverenv}<2>
Note the use of \verb|std::|.
\end{uncoverenv}
\end{frame}


\begin{frame}[fragile]
\frametitle{An Algorithm For Finding Primes Numbers.}
\begin{semiverbatim}
\uncover<1->{\alert<0>{int main (void)}}
\uncover<1->{\alert<0>{\{}}
\uncover<1->{\alert<1>{ \alert<4>{std::}vector<bool> is_prime (100, true);}}
\uncover<1->{\alert<1>{ for (int i = 2; i < 100; i++)}}
\uncover<2->{\alert<2>{ if (is_prime[i])}}
\uncover<2->{\alert<0>{ \{}}
\uncover<3->{\alert<3>{ \alert<4>{std::}cout << i << " ";}}
\uncover<3->{\alert<3>{ for (int j = i; j < 100;}}
\uncover<3->{\alert<3>{ is_prime [j] = false, j+=i);}}
\uncover<2->{\alert<0>{ \}}}
\uncover<1->{\alert<0>{ return 0;}}
\uncover<1->{\alert<0>{\}}}
\end{semiverbatim}
\visible<4->{Note the use of \alert{\texttt{std::}}.}
\end{frame}


% \section{References}
\begin{frame}[allowframebreaks]{References}
\bibliographystyle{apalike}
    \bibliography{ref}
\end{frame}

\section{Acknowledgements}
\begin{frame}
    \centering \wedn \Huge \textbf{Thank you \\ All questions and comments are welcome!}
\end{frame}

\appendix

\section{Appendix I}


\begin{frame}
    
    This is what an appendix would look like.
    
   
\end{frame}

\section{Appendix II} \label{app2}
\begin{frame}
    \begin{block}{Relevant Title}
    sssss
    \end{block}
    
\end{frame}